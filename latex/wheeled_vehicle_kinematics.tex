\section{Wheeled Vehicle Kinematics}
This section is about how coordinate frames move according the geometry, velocities, rotational rates, positions and orientations of their mechanical elements. The focus is for differeny configurations of wheeled vehicles. 

\subsection{Single Wheel}
The most basic mechanical element of wheeled vehicles are their wheels, so let's take a look on how they work from the kinematics point of view.

A wheel is actuated with a motor providing a rotation rate to its axis, $w_w$. This rotation rate, will cause a linear velocity of the wheel center as: 
\begin{equation}
 v_w = w_w r_w
\end{equation}
where $r_w$ is the wheel radius. It is a straightforward example of how the geometry of a body shapes the relation between the actuation variable (rotational rate, $w_w$), and the derived state of the `vehicle` (linear velocity, $v_w$)

In the following subsections the goal will be always to find the relation between the vehicle kinematics state (velocities) and its actuators. 

\subsection{Differential drive}
Differential wheeled configuration is an essential and widely used architecture, due to its building simplicity and good manoeuvring properties, since it allows the platform to turn on the spot. 

Figure xxx shows the configuration as well as the involved parameters and variables. 

Placing the platform axis at the mid point on the baseline joining the wheels, we define the twist of this frame as $(v_x, v_y, w_z)$, and it can be computed as: 


\subsection{Bike}

\subsection{Tricycle}

\subsection{Ackermann (car-like)}

\subsection{Double Ackermann}

\subsection{Omniwheels}


\subsection{Dynamics}
Coriolis: cases: Feature point, people tracking, ...


